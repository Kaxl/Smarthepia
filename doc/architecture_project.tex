\documentclass[a4paper]{article}

% Packages
\usepackage{listings}
\usepackage{color}
\usepackage[utf8]{inputenc}
\usepackage{listingsutf8}
\usepackage{graphicx}
\usepackage{epstopdf}
\usepackage{fancyhdr}
\usepackage[T1]{fontenc}
%\usepackage[top=2cm, bottom=2cm, left=3.5cm, right=2cm]{geometry} % Les marges.
\usepackage[top=2cm, bottom=2cm, left=2cm, right=2cm]{geometry} % Les marges.
\usepackage{hyperref}

\definecolor{mygreen}{rgb}{0,0.6,0}
\definecolor{mygray}{rgb}{0.5,0.5,0.5}
\definecolor{mymauve}{rgb}{0.58,0,0.82}
\definecolor{bggray}{rgb}{0.95, 0.95, 0.95}
\lstset{inputencoding=utf8/latin1}
\lstset{ %
    backgroundcolor=\color{bggray},   % choose the background color; you must add \usepackage{color} or \usepackage{xcolor}
    basicstyle=\footnotesize,        % the size of the fonts that are used for the code
    breakatwhitespace=false,         % sets if automatic breaks should only happen at whitespace
    breaklines=true,                 % sets automatic line breaking
    captionpos=b,                    % sets the caption-position to bottom
    commentstyle=\color{mygreen},    % comment style
    deletekeywords={...},            % if you want to delete keywords from the given language
    escapeinside={\%*}{*)},          % if you want to add LaTeX within your code
    extendedchars=true,              % lets you use non-ASCII characters; for 8-bits encodings only, does not work with UTF-8
    frame=single,                    % adds a frame around the code
    frameround=tttt                  % tttt for having the corner round.
    keepspaces=true,                 % keeps spaces in text, useful for keeping indentation of code (possibly needs columns=flexible)
    keywordstyle=\color{blue},       % keyword style
    language=Matlab,                 % the language of the code
    morekeywords={*,...},            % if you want to add more keywords to the set
    numbers=left,                    % where to put the line-numbers; possible values are (none, left, right)
    numbersep=5pt,                   % how far the line-numbers are from the code
    numberstyle=\tiny\color{mygray}, % the style that is used for the line-numbers
    rulecolor=\color{black},         % if not set, the frame-color may be changed on line-breaks within not-black text (e.g. comments (green here))
    showspaces=false,                % show spaces everywhere adding particular underscores; it overrides 'showstringspaces'
    showstringspaces=false,          % underline spaces within strings only
    showtabs=false,                  % show tabs within strings adding particular underscores
    stepnumber=1,                    % the step between two line-numbers. If it's 1, each line will be numbered
    stringstyle=\color{mymauve},     % string literal style
    tabsize=2,                       % sets default tabsize to 2 spaces
    title=\lstname                   % show the filename of files included with \lstinputlisting; also try caption instead of title
}

% Header
\pagestyle{fancy}
\fancyhead[L]{Axel Fahy \& Rudolf Höhn}
\fancyhead[R]{\today}


\title{Architecture for Smarthepia project\\Distributed systems}
\author{Axel Fahy \& Rudolf Höhn}
\date{\today}


\begin{document}
\maketitle

\section{Overview}
This document talks about the architecture used in Smarthepia project. Now, there are several Rasberry PI deployed over 2 floors at hepia which provide data from sensors that retrieve information on the room temperature, humidity, etc.. \\\\
Our goal is to develop a web based application that displays data from these sensors in a cool way.

\section{Technologies used}
\subsection{Database}
We use MongoDB. The model is :
\begin{verbatim}
{
    "battery": "Number",
    "controller": "String",
    "humidity": "Number",
    "location": "String",
    "luminance": "Number",
    "motion": "Boolean",
    "sensor": "Number",
    "temperature": "Number",
    "updateTime": "Number"
}
\end{verbatim}
\subsection{Client REST}
Its job is to collect data from the Rasberry PI and insert them into the database. The client is implemented using Scala.
\subsection{Server REST}
The role of this server is to provide routes to get the information from the database with filters like between dates and things like this.
The server is also implemented using Scala, but using Play framework.
\subsection{Server Web}
This server is built with NodeJS and it is used to serve the web pages. We choose to separate the server REST and server Web to have the possibility to build other applications using the same API.
\subsection{Front END}
The web application is implemented with AngularJS. Its goal is to provide a clean and userfriendly interface in order to analyze the data from the sensors.

\section{Deployment}
The whole application is deployed on a Switch Engines Virtual Machine. The client REST is running since mid-January and collects data every four minutes on each sensor of the Raspberry Pi 3.\\\\
You can access the web application on \url{http://86.119.33.122:5000} until 1st February (after this day, Switch Engines will probably shut it down).
\end{document}
